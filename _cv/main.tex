\documentclass[10pt,letterpaper]{article} % Use the custom resume.cls style

\usepackage{verbatim} % usefull to coment out blocs
\usepackage{mycv}


\usepackage{mypublicationlist}

\bibliography{refs}

% Define title and author
\title{Curriculum Vitae}
\author{Ant\^onio Horta Ribeiro}


\begin{document}

\maketitle

\begin{tcolorbox}[standard jigsaw, opacityback=0]
    \vspace{-4pt}
\begin{multicols}{2}
    \small
    \textbf{Current Position:}\\
    \, Postdoctoral Fellow\\
    \, Uppsala University \\
    \, Department of Information Technology,\\
    \, Division of Systems and Control\\
    {\bf Work Address:} Room 103146, hus 10 \\
    \phantom{\bf Work address:} Lägerhyddsvägen 1, Uppsala, Sweden\\
    {\bf Postal address:} Box 337 - 751 05, Uppsala, Sweden\\
    {\bf Email:} antonio.horta.ribeiro@it.uu.se\\
    {\bf Website:} antonior92.github.io
\end{multicols}
\end{tcolorbox}

\section*{Academic Positions} % Section title


    \cventry{ Postdoctoral Fellow }
    { Fev. 2021 -   Now  }
    { Department of Information Technology, Uppsala University  }
    { Uppsala, Sweden  }
    { I am working under the supervision of Thomas Schön on the intersection of machine learning, signal processing, and control theory.}

    \cventry{ Postdoctoral Associate }
    { Mar. 2020 -   Fev. 2021  }
    { Department of Computer Science, UFMG  }
    { Belo Horizonte, Brasil  }
    { I worked on developing new machine learning algorithms and studying its application to engineering and health care. My position was funded by the Brasilian Agency CAPES, through the institutional internalization program (PRINT). I funded my own with money obtained from an open call from the Brazilian Agency CAPES.}


\section*{Education} % Section title


    \cventry{ Ph.D., Electrical Engineering }
    { Aug. 2017 - Mar. 2020 }
    { Universidade Federal de Minas Gerais (UFMG) }
    { Brazil }
    { I was supervised by Luis Antonio Aguirre and co-supervised by Thomas B. Schon.  I stayed one year, from Sept. 2018 to Sept. 2019, as a guest doctoral student at Uppsala University (Sweden). My Thesis named "Learning nonlinear differentiable models for signals and systems with applications" won the award of Best thesis in the Electrical Engineering department and also the best thesis in Engineering and Physical Sciences in the University. }

    \cventry{ M.Sc., Electrical Engineering }
    { Jan. 2016 - Jul. 2017 }
    { Universidade Federal de Minas Gerais (UFMG) }
    { Brazil }
    { I was supervised by Luis Antonio Aguirre. My thesis was named "Recurrent Structures in System Identification". I completed 25 credits the equivalent 375 hours in class and my grade pointed average was 5.0 out of 5.0. }

    \cventry{ B.S.E., Electrical Engineering }
    { Jan. 2016 - Jul. 2017 }
    { Universidade Federal de Minas Gerais (UFMG) }
    { Brazil }
    { I completed a total of 240  credits (3600 class-hours). And obtained a grade pointed average 4.91 out of 5.00. My course work included disciplines in control engineering, signal processing, eletrical Drives, power eletronics, system identification, electrical circuits, optimization and communications. }



\section*{Awards}


    \cventry{ Benzelius award }
    { 2022 }
    { Royal Society of Sciences in Uppsala }
    { Sweden }
    { I was awarded the Benzelius Award (Benzeliusbelöningarna) due to my 'contributions to fundamental method development in machine learning and control technology, as well as its use to solve important problems in cardiology'. The prize is awarded yearly by the Royal Society of Sciences in Uppsala (Kungliga Vetenskaps-Societeten i Uppsala): the oldest of the royal academies in Sweden, founded in 1710. Named after Erik Benzelius, the prize is awarded to young researchers and comes with the amount of 25000 kronors. }

    \cventry{ Best Ph.D. Thesis in Engineering and Physical Sciences }
    { 2021 }
    { Universidade Federal de Minas Gerais }
    { Belo Horizonte, Brazil }
    { My Ph.D. thesis was awarded the best Ph.D. thesis defended in 2020 in engineering and physical sciences at the Universidade Federal de Minas Gerais (UFMG), Brazil. In portuguese: Grande Premio de Teses na área de ciências exatas e da terra e engenharias. }

    \cventry{ Best Ph.D. Thesis in Electrical Engineering }
    { 2021 }
    { Universidade Federal de Minas Gerais }
    { Belo Horizonte, Brazil }
    { My thesis was awarded the best Ph.D. thesis defended in 2020 in the Department of Electrical Engineering at the Universidade Federal de Minas Gerais (UFMG), Brazil. The thesis was then forwarded to compete with the thesis from all other Engineering and Physical Sciences departments at the university (where it was also awarded the best thesis, see the award above). }

    \cventry{ Young Author Award (Honorable Mention) }
    { 2021 }
    { 19th IFAC Symposium on System Identification }
    { Online }
    { I have been one of the three finalists of the Young Author Award with the paper `Beyond Occam’s Razor in System Identification:  Double-Descent when Modeling Dynamics'. }

    \cventry{ Best Poster Award }
    { 2019 }
    { SciLifeLab Science Summit }
    { Uppsala, Sweden }
    { I have been awarded the best poster award for the work `Automatic Diagnosis of Short-Duration 12-Lead ECG using a Deep Convolutional Network'. }

    \cventry{ Travel Award }
    { 2018 }
    { Machine Learning for Health (ML4H) Workshop at NeurIPS }
    { Montreal, Canada }
    { I have been awarded the travel award for the work `Automatic Diagnosis of Short-Duration 12-Lead ECG using a Deep Convolutional Network' and had my expenses covered by the award. }


\section*{Scholarships}


    \cventry{ SFVE-A mobility grant }
    { 2023 }
    { French Institute of Sweden }
    { Sweden }
    { I have been granted the Svensk Fransk Vetenskap–Anslag (SFVE-A) grant. }

    \cventry{ ELISE mobility grant }
    { 2023 }
    { European Network of AI Excellence Centres }
    { Europe }
    { I have been granted for a research visit to Francis Bach group at ENS/INRIA during Spring 2023 }

    \cventry{ CAPES-PRINT }
    { 2020-2021 }
    { CAPES }
    { Brazil }
    { I have been granted a scholarship from the Brasilian Agency CAPES for internacionalization. }

    \cventry{ Split-site Ph.D. Scholarship }
    { 2019 }
    { CNPq }
    { Brazil }
    { I have been granted a scholarship from the Brasilian Agency CNPq for staying one year of my Ph.D. in Uppsala University, Sweden. }

    \cventry{ Ph.D. Scholarship }
    { 2018-2020 }
    { CNPq }
    { Brazil }
    { I have been granted a scholarship from the Brasilian Agency CNPq during my doctoral studies. }

    \cventry{ M.S. Scholarship }
    { 2016-2017 }
    { CAPES }
    { Brazil }
    { I have been granted a scholarship from the Brasilian Agency CAPES during my master studies. }


\section*{Supervision}


\subsection*{\noindent Ph.D. students, co-supervisor  }
    
        \cventry{  Daniel Gedon  }
        { Aug. 2019 - Aug. 2024 (estimated) }
        {  }
        { Uppsala University, Sweden }
        { { Disentangled Representation Learning in Self-Supervised Models } }
        

\subsection*{\noindent M.Sc. students, supervisor  }
    
        \cventry{  Oscar Larsson  }
        { Feb. 2022 - July 2022 }
        {  }
        { Uppsala University, Sweden }
        { { Generation and Detection of Adversarial Attacks in the Power Grid } }
        
        \cventry{  Theogene Habineza  }
        { Jan. 2022 - June 2022 }
        {  }
        { Uppsala University, Sweden }
        { { Deep Learning-Based Risk Prediction of Atrial Fibrillation Using the 12-lead ECG } }
        

\subsection*{\noindent M.Sc. students, subject reviewer  }
    
        \cventry{  Johan Millberg  }
        { Jan. 2023 - June 2023 (estimated) }
        {  }
        { Uppsala University, Sweden }
        { { Evaluating clusters of financial time series } }
        
        \cventry{  Christie Courtnage  }
        { Jan. 2022 - June 2022 }
        {  }
        { Uppsala University, Sweden }
        { { An extension to Semi-Supervised Learning using Shapley Value Data Valuation } }
        
        \cventry{  Meenal Pathak  }
        { Feb. 2022 - Apr. 2022 }
        {  }
        { Uppsala University, Sweden }
        { { Automated Accounting using Machine Learning } }
        


\section*{Longer scientific visits}

    
        \cventry{ Ecolé Normale Superiore / INRIA }
        { March 2023 - June 2023 }
        { Visiting researcher }
        { Paris, France }
        { I was hosted by Francis Bach at the SIERRA team.}
    
        \cventry{ Uppsala University }
        { September 2018 - September 2019 }
        { Visiting PhD student }
        { Uppsala, Sweden }
        { I was hosted by Thomas B. Schön at the Department of Information Technology.}
    

\section*{Teaching}


    \cventry{ Advanced Probabilistic Machine Learning  }
    {   Fall - 2022  }
    { Uppsala University, Sweden }
    { MSc and PhD level, Course responsible and lecturer }
    { Course details: 90 MSc (+11 PhD) students, 5 + 2.5 credits.  \emph{ I was the main responsible for the course. I was involved in lecturing and in preparing the final exam. I also updated the course structure, lecture content and added exercises based on previous year feedback. } }
    
    \cventry{ Artificial Intelligence and Machine Learning  }
    {   Spring - 2022  }
    { WASP Graduate School, Sweden }
    { PhD level, Teaching assistant }
    { Course details: 94 students, 6 credits.  \emph{ I was responsible for the design of the course assignment. } }
    
    \cventry{ Advanced Probabilistic Machine Learning  }
    {   Fall - 2021  }
    { Uppsala University, Sweden }
    { MSc level, Lecturer }
    { Course details: 125 MSc (+4 PhD) students, 5 + 2.5 credits.  \emph{ I was involved in lecturing and in the preparation of the exam. } }
    
    \cventry{ The unreasonable effectiveness of overparameterized machine learning models  }
    {   Fall - 2021  }
    { Uppsala University, Sweden }
    { MSc and PhD level, Course developer and organizer }
    { Course details: 13 students, 3 credits.  \emph{ I was the main responsible for the development of and organization of this new seminar course. I was involved in the choice of papers, in leading the discussion and was responsible for preparing all the assignments. } }
    
    \cventry{ Deep Learning  }
    {   Spring - 2021  }
    { Uppsala University, Sweden }
    { PhD level, Teaching assistant }
    { Course details: 54 students, 5 + 3 credits.  \emph{ I was responsible for in preparing the assignment. } }
    
    \cventry{ Engenharia de Controle (Control Engineering)  }
    {   2nd - 2016  }
    { Universidade Federal de Minas Gerais, Brazil }
    { BSc level, Teaching assistant }
    { Course details: 50 students, 6 credits.  \emph{ I was responsible for exercise sections and developing the assignment. Also, I was involved in preparing the exam. } }
    
    \cventry{ Controle Digital  (Digital Control)  }
    {   2nd - 2016  }
    { Universidade Federal de Minas Gerais, Brazil }
    { BSc level, Teaching assistant }
    { Course details: 40 students, 4 credits.  \emph{ I was responsible for exercise sections and developing the assignment. Also, I was involved in preparing the exam. } }
    




\section*{Professional activity}

\subsection*{Peer reviewing: journal papers}

 {\em IEEE Transactions on Automatic Control } (2021),  {\em Heart } (2021),  {\em IEEE Transactions on Instrumentation and Measurement } (2021),  {\em International Journal of System Science } (2021),  {\em Proceedings of the National Academy of Sciences (PNAS) } (2020),  {\em Automatica } (2020),  {\em IEEE Transactions on Biomedical Engineering } (2020),  {\em IEEE Control Systems Letters (L-CSS) } (2020),  {\em Systems and Control Letters } (2020),  {\em Chaos, Solutions and Fractals } (2020),  {\em Chest } (2020),  {\em Journal of Electrocardiology } (2020),  {\em Journal of Control, Automation and Electrical Systems } (2015-2018), 

\subsection*{Peer reviewing: conference papers}

 {\em Learning for Dynamics and Control (L4DC) } (2022),  {\em International Conference on Artificial Intelligence and Statistics (AISTATS) } (2022),  {\em IFAC Symposium on System Identification (SysId) } (2021),  {\em Learning for Dynamics and Control (L4DC) } (2021),  {\em European Control Conference (ECC) } (2021),  {\em IEEE Conference on Decision and Control (CDC) } (2020),  {\em IFAC World Conference } (2020),  {\em American Control Conference } (2018),  {\em International Conference on Modelling, Identification and Control } (2017),  {\em IFAC World Conference } (2017), 

\subsection*{Expert assignments}

 ELLIS (European Laboratory for Learning and Intelligent Systems) PhD Program: Recruitment evaluator \hfill {\em 2020 } \\  Co-chair at the session `Parameter Estimation 1' at the 19th IFAC Symposium on System Identification \hfill {\em 2021 } \\ 

\subsection*{External examiner in Ph.D. and M.Sc. defenses}


    \cventry{  Najmeh Fayyazifar , Level: Ph.D. }
    { 2022 }
    {  }
    { Edith Cowan University, Australia }
    { {\it Deep learning and neural architecture search for cardiac arrhythmias classification } }

    \cventry{  Thiago de Almeida Ushikoshi , Level: M.Sc. }
    { 2022 }
    {  }
    { Universidade Federal de Minas Gerails, Brazil }
    { {\it Learning Nonlinear Dynamics With Echo State Networks } }



\subsection*{Invited talks}

\subsubsection*{Universities}

    
        \shortcventry{ INRIA, Paris @ SIERRA team  }
    { March 2023 }
    { Overparametrized Linear Regression under Adversarial Attacks  }
    { }{}
    

    
        \shortcventry{ Seminars on Advances in Probabilistic Machine Learning @ Aalto University and ELLIS unit Helsinki  }
    { November 2022 }
    { Adversarial Attacks in Linear Regression  }
    { }{}
    

    
        \shortcventry{ University of British Columbia, Canada @ Christos Thrampoulidis group (Online)  }
    { June 2022 }
    { Overparameterized Linear Regression under Adversarial Attacks  }
    { }{}
    

    
        \shortcventry{ University of Luxembourg @ Systems Control Group, LCSB (Online)  }
    { March 2022 }
    { Deep Neural Networks for Automatic ECG Analysis  }
    { }{}
    

    
        \shortcventry{ Techinion, Israel @  AIMLab group (Online)  }
    { March 2021 }
    { Artificial intelligence for ECG classification and prediction of the risk of death  }
    { }{}
    


\subsubsection*{Conferences}


    
        \shortcventry{ International Congress on Electrocardiology (Online)  }
    { April 2021 }
    { Artificial intelligence for ECG classifcation and prediction of the risk of death  }
    { }{}
    



\subsection*{Open source contributions}

{\bf Scipy team member} \hfill {\em 2017 - 2021} \\
I was one of the SciPy development team members.
\href{https://www.scipy.org}{SciPy} is one of the core scientific libraries in Python and I was invited to the core team
for having contributed with the implementation of signal filters
and optimization method.
My GitHub account: \href{https://github.com/antonior92}{https://github.com/antonior92}
contain a complete list of my open-source contributions.

\section*{Additional work experience} % Section title


    \cventry{ Software Developer }
    { May. 2017 -   Aug. 2017 }
    { Google Summer of Code }
    { Scipy }
    { I have successfully completed Google Summer of Code program under the mentorship of Matt Haberland, Nikolay Mayorov and Ralf Gommers. My project was the implementation of an interior-point solver for large-scale nonlinear programming problems. The result is the method trust-contr, now openly available as part of the open source scientific library SciPy, in Python. }

    \cventry{ Hardware Team Intern }
    { Jan. 2015 -   Dec. 2015 }
    { Invent Vision }
    { Belo Horizonte, Brazil }
    { I was part of the hardware development team and worked designing FPGA-based cameras. The major project I have worked on while there was the design and implementation of a stereo camera. }

    \cventry{ Undergraduate Researcher }
    { Jun. 2013 -   Jan. 2015 }
    { Research and development project with Petrobras Oil Company, UFMG }
    { Belo Horizonte, Brazil }
    { I worked on the development of methods for identification of oil well mathematical models under the supervision of Professor Luis Antonio Aguirre. My position was funded by the Petrobras Oil Company through the Christiano Ottoni Foundation (FCO) in the modality bolsa de iniciação científica. }



\section*{Additional education}


    \cventry{ Mini-course on Nonlinear System Identification }
    {\em  2019 }
    { Eindhoven University of Technology }
    { The Netherlands }
    { I took part on the 3 days mincourse on nonlinear system identification to take place on Eindhoven University of Technology. }

    \cventry{ Probabilistic Graphical Models Specialization }
    {\em  2018 }
    { Coursera (Stanford) }
    { Online }
    { I have successfully completed the 3 online courses about probabilistic graphical models, titled `Representation', `Inference', `Learning'. }

    \cventry{ Deep Learning Specialization }
    {\em  2018 }
    { Coursera (deeplearning.ai) }
    { Online }
    { I have successfully completed the 5 online courses about deep learning offered in Coursera, `Neural Networks and Deep Learning', `Improving Deep Neural Networks: Hyperparameter tuning, Regularization and Optimization', `Structuring Mechine Learning Projects', `Convolutional Neural Networks', `Sequence Models'. }


\section*{Languages}

Portuguese (mother tongue)\\
English (fluent)\\
Spanish (intermediate knowledge)\\
Swedish (elementary knowledge)

\subsubsection*{Language certificates}

Certificate in Advanced English (Council of Europe Level C1) - Cambridge English Language Assessment, 2014


\pagenumbering{roman}

% Reset page
\setcounter{page}{1}

 \begin{center}%
  \let \footnote \thanks
    {\LARGE \bf Publication List \par}%
  \end{center}
  \par
  \vskip 2em




\subsection*{Bibliometrics}
\begin{tcolorbox}[standard jigsaw, opacityback=0]
    \vspace{-4pt}
    \begin{multicols}{2}
    \footnotesize

\textbf{Journal Publications:} 15\\
\textbf{Conference Publications:} 12\\
\textbf{Workshop papers:} 7\\
\textbf{Theses:} 3\\

 \noindent
 \textbf{Citations:} 16\,673\\
 \textbf{h-index:} 11\\
 \textbf{i10-index:} 11\\
 \emph{According to Google Scholar (2023-02-22)}
\end{multicols}
\vspace{-20pt}
\end{tcolorbox}

\subsection*{Scientific Database Identifiers}

\begin{tcolorbox}[standard jigsaw, opacityback=0]
    \vspace{-4pt}
    \footnotesize
{\bf ORCID}:  \href{https://orcid.org/0000-0003-3632-8529}{0000-0003-3632-8529}\\
{\bf DBLP}: \href{https://dblp.org/pid/202/1699.html}{202/1699}\\
{\bf SCOPUS ID}: \href{https://www.scopus.com/authid/detail.uri?authorId=57191699148}{57191699148}
  \\
{\bf Google Scholar}: \href{https://scholar.google.com.br/citations?user=5t_sZdMAAAAJ}{Antonio H. Ribeiro}

\end{tcolorbox}



\printmypublications

\end{document}

\newpage

\section*{Selected publications}

Bellow the list of my five most relevant publications to the moment with contribution and relevance statement.






\end{document}

\newpage





\begin{center}
{\LARGE \bf Additional information \par}
\end{center}
\vspace{20pt}

\section*{Contact References}

I include here the contact information of two references: Luis Antonio Aguirre was my supervisor during my Ph.D.
Thomas Schön is supervising me through my postdoc and is the leader of the group I am now in.
He also hosted me in Uppsala as visiting student during one year of my Ph.D.


\vspace{12pt}
\begin{center}
\begin{tcolorbox}[width=6in, standard jigsaw, opacityback=0]
    \vspace{-4pt}
    \footnotesize
    {\bf Name}: Thomas B. Sch\"on  \\
    {\bf Instituition}: Uppsala University (Sweden) \\
    {\bf Position}: Professor at the Department of Information Technology.\\
    {\bf Email}: thomas.schon@it.uu.se
\end{tcolorbox}
\begin{tcolorbox}[width=6in, standard jigsaw, opacityback=0]
    \vspace{-4pt}
    \footnotesize
    {\bf Name}: Luis Antonio Aguirre \\
    {\bf Instituition}: Federal University of Minas Gerais (Brazil) \\
    {\bf Position}: Professor at the Electronic Engineering Department.\\
    {\bf Email}: aguirre@ufmg.br; aguirre.zanon@gmail.com
\end{tcolorbox}
\end{center}

\vspace{28pt}



\end{document}